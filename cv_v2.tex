\documentclass[12pt,letterpaper]{article}

\usepackage{hyperref}
\usepackage{geometry}
\usepackage[T1]{fontenc}
\usepackage{natbib}

% name here
\def\name{\textbf{Adrian M. Price-Whelan}}

% Date formatting.
\usepackage[yyyymmdd]{datetime}
\renewcommand{\dateseparator}{-}

% ADS query link
\def\adsurl{http://adsabs.harvard.edu/cgi-bin/nph-abs_connect?db_key=AST&db_key=PRE&qform=AST&arxiv_sel=astro-ph&arxiv_sel=cond-mat&arxiv_sel=cs&arxiv_sel=gr-qc&arxiv_sel=hep-ex&arxiv_sel=hep-lat&arxiv_sel=hep-ph&arxiv_sel=hep-th&arxiv_sel=math&arxiv_sel=math-ph&arxiv_sel=nlin&arxiv_sel=nucl-ex&arxiv_sel=nucl-th&arxiv_sel=physics&arxiv_sel=quant-ph&arxiv_sel=q-bio&sim_query=YES&ned_query=YES&adsobj_query=YES&aut_logic=OR&obj_logic=OR&author=price-whelan\%2C+Adrian&object=&start_mon=&start_year=&end_mon=&end_year=&ttl_logic=OR&title=&txt_logic=OR&text=&nr_to_return=200&start_nr=1&jou_pick=ALL&ref_stems=&data_and=ALL&group_and=ALL&start_entry_day=&start_entry_mon=&start_entry_year=&end_entry_day=&end_entry_mon=&end_entry_year=&min_score=&sort=SCORE&data_type=SHORT&aut_syn=YES&ttl_syn=YES&txt_syn=YES&aut_wt=1.0&obj_wt=1.0&ttl_wt=0.3&txt_wt=3.0&aut_wgt=YES&obj_wgt=YES&ttl_wgt=YES&txt_wgt=YES&ttl_sco=YES&txt_sco=YES&version=1}

% PDF metadata
\hypersetup{
  colorlinks = true,
  urlcolor = [rgb]{0.1,0.25,0.5},
  pdfauthor = {\name},
  pdfkeywords = {astrophysics, astronomy, physics},
  pdftitle = {\name: Curriculum Vitae},
  pdfsubject = {Curriculum Vitae},
  pdfpagemode = UseNone
}

% page size
\geometry{
  body={6.5in, 9.0in},
  left=1.0in,
  top=1.0in
}

% text formatting
\usepackage{color}
\definecolor{grey}{gray}{0.5}
\newcommand{\deemph}[1]{\textcolor{grey}{\footnotesize{#1}}}

% heading / footing
\usepackage{fancyheadings}
\pagestyle{fancy}
\renewcommand{\headrulewidth}{0pt}
\lhead{\deemph{Adrian M. Price-Whelan}}
\chead{\deemph{Curriculum Vitae}}
\rhead{\deemph{\thepage}}
\cfoot{\deemph{Last updated: \today}}

% Don't indent paragraphs.
\setlength\parindent{0em}

% Make lists without bullets and compact spacing
\renewenvironment{itemize}{
  \begin{list}{}{
    \setlength{\leftmargin}{1em}
    \setlength{\itemsep}{0.em}
    \setlength{\parskip}{0pt}
    \setlength{\parsep}{0.25em}
    % \setlength{\itemindent}{0em}
  }
}{
  \end{list}
}

% Change section font size and spacing
\usepackage{titlesec}
\titleformat{\section}{\normalfont\fontsize{14pt}{0}\bfseries}{\thesection}{}{}
\titleformat{\subsection}{\normalfont\fontsize{12pt}{0}\bfseries}{\thesubsection}{}{}
\titlespacing{\section}{0em}{-0.em}{0.25em}
\titlespacing{\subsection}{0.5em}{0.em}{0.25em}

% literature links (thanks @dfm)
\newcommand{\doi}[2]{\emph{\href{http://dx.doi.org/#1}{{#2}}}}
\newcommand{\ads}[2]{\href{http://adsabs.harvard.edu/abs/#1}{{#2}}}
\newcommand{\arxiv}[1]{{\href{http://arxiv.org/abs/#1}{arXiv:{#1}}}}

% Journal names
\newcommand{\aanda}{A\&A}
\newcommand{\aj}{AJ}
\newcommand{\apj}{ApJ}
\newcommand{\apjs}{ApJS}
\newcommand{\apjl}{ApJL}
\newcommand{\pasp}{PASP}
\newcommand{\mnras}{MNRAS}
\newcommand{\mnrasl}{MNRAS Letters}


\begin{document}\thispagestyle{empty}\sloppy\sloppypar

% Name and contact, website, etc.
{\huge \name}
\vspace{-0.25em}

\begin{itemize}
  \item The Cosmic Dawn Center, Niels Bohr Institute, University of Copenhagen
  \item  ORCID ID: \href{https://orcid.org/0000-0001-9058-3892}{0000-0001-9058-3892}
  \item Email: \href{mailto:jselsing@nbi.ku.dk}{jselsing@nbi.ku.dk} 
  %\item Personal website: \href{http://astroman.io}{astroman.io} 
%  --- \href{http://somename.nn}{http://somename.nn}
\end{itemize}


\section*{Personal data}

\begin{itemize}
	\item Nationality: Danish
	\item Spouse: Malene Selsing Mouritzen
	\item Child: Marie Selsing (b. 2016) 
	\item Paternity leave: Jan. - Jul. 2017
	
\end{itemize}


\section*{Employment}
\begin{itemize}
	\item Post-doc 2018, The Cosmic Dawn Center, Niels Bohr Institute, University of Copenhagen.  2018 --
\end{itemize}


\section*{Education}
	\begin{itemize}
	\item PhD 2018, Astronomy, Dark Cosmology Centre, University of Copenhagen.
		{Advisor: L. Christensen}
	\item MA 2015, Physics, University of Copenhagen.
		{Advisor: L. Christensen}
	\item BA 2011, Physics, University of Copenhagen.
		{Advisor: J. Hjorth}
	\end{itemize}


\section*{Core scientific  fields}

\begin{itemize}
	\item Kilonovae, Supernovae, Gamma-Ray Bursts, Tidal Disruption Events, Quasars
\end{itemize}


\section*{International collaborations}

\begin{itemize}
	\item Electromagnetic counterparts of gravitational wave sources at the Very Large Telescope (\href{http://www.engrave-eso.org}{ENGRAVE}), Swift response Transient Astrophysics Research (STARGATE), X-Shooter Gamma-Ray Burst collaboration (XSGRB)
\end{itemize}


\section*{Awards}

\begin{itemize}
	\item Jens Martin prize, awarded for excellence in teaching (2015)
\end{itemize}


\section*{Teaching}

\begin{itemize}
	\item \emph{Electromagnetic counterparts of gravitational wave sources data training}, Invited tutor, 2018, Bertinoro, Italy
	\item \emph{Introductory Cosmology}, TA, 2013, University of Copenhagen
	\item \emph{Introductory Classical Mechanics}, TA, 2013, University of Copenhagen
	\item \emph{Advanced Classical Mechanics}, TA, 2013, University of Copenhagen
\end{itemize}

\section*{Public outreach}

\begin{itemize}
	\item I make an effort to speak to the general public as often as possible, usually a few times a month. Additionally, I have on several occasions appeared on danish national television and in most major newspapers, been cited in 52 individual articles on the observations of GW170817/GRB170817A

    
\end{itemize}

\section*{Open source development (\href{https://github.com/jselsing}{GitHub profile})}


\end{document}
