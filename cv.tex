\documentclass[12pt,letterpaper]{article}

\usepackage{hyperref}
\usepackage{geometry}
\usepackage[T1]{fontenc}
\usepackage{natbib}
\usepackage{mdwlist}
% name here
\def\name{\textbf{Jonatan Selsing}}

% Date formatting.
\usepackage[yyyymmdd]{datetime}
\renewcommand{\dateseparator}{-}

% ADS query link
\def\adsurl{http://adsabs.harvard.edu/cgi-bin/nph-abs_connect?db_key=AST&db_key=PRE&qform=AST&arxiv_sel=astro-ph&arxiv_sel=cond-mat&arxiv_sel=cs&arxiv_sel=gr-qc&arxiv_sel=hep-ex&arxiv_sel=hep-lat&arxiv_sel=hep-ph&arxiv_sel=hep-th&arxiv_sel=math&arxiv_sel=math-ph&arxiv_sel=nlin&arxiv_sel=nucl-ex&arxiv_sel=nucl-th&arxiv_sel=physics&arxiv_sel=quant-ph&arxiv_sel=q-bio&sim_query=YES&ned_query=YES&adsobj_query=YES&aut_logic=OR&obj_logic=OR&author=selsing\%2C+Jonatan&object=&start_mon=&start_year=&end_mon=&end_year=&ttl_logic=OR&title=&txt_logic=OR&text=&nr_to_return=200&start_nr=1&jou_pick=ALL&ref_stems=&data_and=ALL&group_and=ALL&start_entry_day=&start_entry_mon=&start_entry_year=&end_entry_day=&end_entry_mon=&end_entry_year=&min_score=&sort=SCORE&data_type=SHORT&aut_syn=YES&ttl_syn=YES&txt_syn=YES&aut_wt=1.0&obj_wt=1.0&ttl_wt=0.3&txt_wt=3.0&aut_wgt=YES&obj_wgt=YES&ttl_wgt=YES&txt_wgt=YES&ttl_sco=YES&txt_sco=YES&version=1}

% PDF metadata
\hypersetup{
  colorlinks = true,
  urlcolor = [rgb]{0.1,0.25,0.5},
  pdfauthor = {\name},
  pdfkeywords = {astrophysics, astronomy, physics},
  pdftitle = {\name: Curriculum Vitae},
  pdfsubject = {Curriculum Vitae},
  pdfpagemode = UseNone
}

% page size
\geometry{
  body={6.5in, 9.0in},
  left=1.0in,
  top=1.0in
}

% text formatting
\usepackage{color}
\definecolor{grey}{gray}{0.5}
\newcommand{\deemph}[1]{\textcolor{grey}{\footnotesize{#1}}}

% heading / footing
\usepackage{fancyheadings}
\pagestyle{fancy}
\renewcommand{\headrulewidth}{0pt}
\lhead{\deemph{Jonatan Selsing}}
\chead{\deemph{Curriculum Vitae}}
\rhead{\deemph{\thepage}}
\cfoot{\deemph{Last updated: \today}}

% Don't indent paragraphs.
\setlength\parindent{0em}

% Make lists without bullets and compact spacing
\renewenvironment{itemize}{
  \begin{list}{}{
    \setlength{\leftmargin}{1em}
    \setlength{\itemsep}{0.em}
    \setlength{\parskip}{0pt}
    \setlength{\parsep}{0.25em}
    % \setlength{\itemindent}{0em}
  }
}{
  \end{list}
}

% Change section font size and spacing
\usepackage{titlesec}
\titleformat{\section}{\normalfont\fontsize{15pt}{0}\bfseries}{\thesection}{}{}
\titleformat{\subsection}{\normalfont\fontsize{12pt}{0}\bfseries}{\thesubsection}{}{}
\titlespacing{\section}{0em}{-0.em}{0.5em}
\titlespacing{\subsection}{0.5em}{0.em}{0.5em}

% literature links (thanks @dfm)
\newcommand{\doi}[2]{\emph{\href{http://dx.doi.org/#1}{{#2}}}}
\newcommand{\ads}[2]{\href{http://adsabs.harvard.edu/abs/#1}{{#2}}}
\newcommand{\arxiv}[1]{{\href{http://arxiv.org/abs/#1}{arXiv:{#1}}}}

% Journal names
\newcommand{\aanda}{A\&A}
\newcommand{\aj}{AJ}
\newcommand{\apj}{ApJ}
\newcommand{\apjs}{ApJS}
\newcommand{\apjl}{ApJL}
\newcommand{\pasp}{PASP}
\newcommand{\mnras}{MNRAS}
\newcommand{\mnrasl}{MNRAS Letters}
\newcommand{\nature}{\textbf{Nature}}
\newcommand{\natureast}{\textbf{Nature Astronomy}}

\begin{document}\thispagestyle{empty}\sloppy\sloppypar

% Name and contact, website, etc.
{\huge \name}
\vspace{-0.25em}

\begin{itemize}
  \item PhD Student
  \item Dark Cosmology Centre, Niels Bohr Institute, University of Copenhagen
  \item  ORCID ID: \href{https://orcid.org/0000-0001-9058-3892}{0000-0001-9058-3892}
  \item \href{mailto:jselsing@dark-cosmology.dk}{jselsing@dark-cosmology.dk} 
%  --- \href{http://somename.nn}{http://somename.nn}
\end{itemize}


\section*{Personal data}

\begin{itemize}
	\item Nationality: Danish
	\item Spouse: Malene Selsing Mouritzen
	\item Child: Marie Selsing (b. 2016) 
	\item Paternity leave: Jan. - Jul. 2017
	
\end{itemize}


\section*{Education}
	\begin{itemize}
	\item PhD 2018, Astronomy, University of Copenhagen.
		{Advisor: L. Christensen}
	\item MA 2015, Physics, University of Copenhagen.
		{Advisor: L. Christensen}
	\item BA 2011, Physics, University of Copenhagen.
		{Advisor: J. Hjorth}
	\end{itemize}


\section*{Awards}

	\begin{itemize}
    \item Jens Martin prize, awarded for excellence in teaching (2015)
	\end{itemize}

\newpage

\section*{Publications  by topic (\href{\adsurl}{ADS}) (16-06-2018)}
    submitted: 25 ---
    first author: 3 ---
    citations: 655


    \vspace{1em}

	\subsection*{\textbf{Gravitational wave counterparts}}
	\begin{enumerate}

\item Watson et al.,
	{\it Discovery of neutron-capture elements in a neutron star merger},
Submitted to \nature, 2018

\item Pian et al.,
	\doi{10.1038/nature24298}{Spectroscopic identification of r-process nucleosynthesis in a double neutron-star merger},
	\nature, 551, 67, 2017 (\arxiv{1710.05858})

\item Abbot et al.,
	\doi{10.3847/2041-8213/aa91c9}{Multi-messenger Observations of a Binary Neutron Star Merger},
	\apjl, 848, L12, 2017 (\arxiv{1710.05833})

%    \end{enumerate}
\suspend{enumerate}
\subsection*{Gamma-ray bursts}
\resume{enumerate}


\item de Ugarte Postigo et al.,
{\it X-shooter and ALMA spectroscopy of GRB 161023A.
	A study of metals and molecules in the line-of-sight towards a luminous GRB},
submitted to \aanda

\item Zafar et al.,
{\it The 2175 \AA~extinction feature in the optical afterglow spectrum of GRB 180325A at z = 2.25},
accepted for publication in \apjl (\arxiv{1806.00293})

\item Heintz et al.,
	{\it Chemical evolution of high redshift GRB host galaxies},
	submitted to \aanda

\item Zafar et al.,
	{\it X-shooting GRBs at high redshift: probing dust production history},
	submitted to \mnras

\item Tanvir et al.,
	{\it The escape fraction of ionizing radiation from massive stars},
	submitted to \mnras (\arxiv{1805.07318})

\item Zafar et al.,
	\doi{ 10.1093/mnras/sty1380}{VLT/X-shooter GRBs: Individual extinction curves of star-forming regions},
	accepted for publication in \mnras (\arxiv{1805.07016})


\item Heintz et al.,
	\doi{10.1093/mnras/sty1447}{Highly-ionized metals as probes of the circumburst gas in the natal regions of gamma-ray bursts},
	accepted for publication in \mnras (\arxiv{1806.01296})


\item \textbf{Selsing} et al.,
	{\it The X-shooter GRB afterglow legacy sample (XS-GRB)},
	submitted to \aanda (\arxiv{1802.07727})


\item Heintz et al.,
\doi{10.1093/mnras/stx2895}{	
	The luminous, massive and solar metallicity galaxy hosting the Swift $\gamma$-ray burst GRB 160804A at z = 0.737},
	\mnras, 474, 2, 2018, (\arxiv{1711.02706})

\item \textbf{Selsing} et al.,
	{\it The host galaxy of the short GRB111117A at z=2.211: impact on the short GRB redshift distribution and progenitor channels},
	accepted for publication in \aanda (\arxiv{1707.01452})

\item Christensen et al.,
	\doi{10.1051/0004-6361/201731382}{Solving the conundrum of intervening strong MgII absorbers towards GRBs and quasars},
	\aanda, 608A, 84C, 2017, (\arxiv{1709.01084})

\item Heintz et al.,
    \doi{10.1051/0004-6361/201730702}{	
    	Steep extinction towards GRB 140506A reconciled from host galaxy observations: Evidence that steep reddening laws are local},
    \aanda, 601, A83, 2017, (\arxiv{1703.07109})

\item Kr\"uhler et al.,
	\doi{10.1051/0004-6361/201425561}{GRB hosts through cosmic time. VLT/X-Shooter emission-line spectroscopy of 96 γ-ray-burst-selected galaxies at 0.1 <z < 3.6},
	\aanda, 581, A125, 2017, (\arxiv{1505.06743})

\item Japelj et al.,
	\doi{10.1051/0004-6361/201525665}{Spectrophotometric analysis of gamma-ray burst afterglow extinction curves with X-Shooter},
	\aanda, 579, A74, 2017, (\arxiv{1503.03623})

%    \end{itemize}
\suspend{enumerate}
\subsection*{Supernovae}
\resume{enumerate}


\item Anderson et al.,
{\it A nearby superluminous supernova with a long pre-maximum 'plateau' and strong C ii features},
	submitted to \aanda

\item Cano et al.,
	\doi{10.1093/mnras/stx2624}{A spectroscopic look at the gravitationally lensed type Ia SN 2016geu at z=0.409},
	\mnras, 473, 3, 2018, (\arxiv{1708.05534})

\item Rodney et al.,
	\doi{10.1038/s41550-018-0405-4}{Two Peculiar Fast Transients in a Strongly Lensed Host Galaxy},
	\natureast, 2, 324, 2018 (\arxiv{1707.02434})

\item Kelly et al.,
	\doi{10.1038/s41550-018-0430-3}{An individual star at redshift 1.5 extremely magnified by a galaxy-cluster lens},
	\natureast, 2, 334, 2018 (\arxiv{1706.10279})
	
\item Kelly et al.,
	\doi{10.3847/0004-637X/831/2/205}{SN Refsdal: Classification as a Luminous and Blue SN 1987A-like Type II Supernova},
	\apj, 831, 205, 2016, (\arxiv{1512.09093})

\item Rodney et al.,
	\doi{10.3847/0004-637X/820/1/50}{SN Refsdal: Photometry and Time Delay Measurements of the First Einstein Cross Supernova},
	\apj, 820 50, 2016, (\arxiv{1512.05734})

\item Kelly et al.,
	\doi{10.3847/2041-8205/819/1/L8}{Deja Vu All Over Again: The Reappearance of Supernova Refsdal},
	\apjl, 819 L8, 2016, (\arxiv{1512.04654})



%	\end{enumerate}


\suspend{enumerate}
\subsection*{Quasars}
\resume{enumerate}
	
\item \textbf{Selsing} et al.,
	\doi{10.1051/0004-6361/201527096}{An X-Shooter composite of bright 1 < z < 2 quasars from UV to infrared},
	\aanda, 585, A87, 2016, (\arxiv{1510.08058})

\end{enumerate}

\newpage

\section*{Approved observing proposals}

	\begin{itemize}
    \item {\it Piercing dark lenses with highly magnified transients}, ESO/VLT X-shooter,  099.A-0801(A)
    \item {\it Exploiting the extreme magnification of a caustic-crossing event. Imaging a single star at z = 1.49.}, ESO/VLT FORS,  297.A-5026(A)
    \item {\it The wonders of gravitational lensing: Imaging a single star at z = 1.5.}, NOT ALFOSC,  53-407
	\end{itemize}

\section*{Observing experience}

\begin{itemize}
	\item NOAO, Cerro Tololo, Blanco 4m, Cluster weak lensing observations with DECam over 5 half-nights(PI: Anja von der Linden, KIPAC)
	\item Roque de los Muchachos, NOT 2.5m, ALFOSC for variability studies over 2 full nights(PI: Johan Fynbo, DARK).
\end{itemize}


\section*{Scientific Talks}

\begin{itemize}
	\item \emph{The X-shooter sample of GRB afterglows (XSGRB)}, EWASS, Liverpool, 2018
	\item \emph{The spectroscopic evolution of AT2017gfo as seen by VLT/X-shooter and the implications for kilonova models.}, EWASS, Liverpool, 2018
    \item \emph{The X-shooter GRB afterglow sample}, XSGRB workshop, Meudon Observatory - Observatoire de Paris, 2016
    \item \emph{The XSGRB sample}, DARK cake talks, Dark Cosmology Centre, 2016
    \item \emph{A composite spectrum of GRB afterglows}, IMPS takes Esalen, California, 2016 
    \item \emph{Lessons from a bright quasar composite}, Department lunch talks, UC Berkeley, 2016 
    \item \emph{The last rise of Refsdal}, IMPS talk, UC Santa Cruz, 2015
    \item \emph{Quasar composite from UV to near-infrared}, IMPS talk, UC Santa Cruz, 2015
    \item \emph{Local environments of SNe Ic and Ic-BL}, XXIX IAU General Assembly, Hawaii, 2015.
    \item \emph{Composite spectrum of the afterglow sample}, XSGRB workshop, IAA-CSIC Granada, 2014
    \item \emph{IFS of Ic-BL SNe}, Galaxies meet GRBs, Cabo de Gata, 2013


\end{itemize}


\section*{Open source development (\href{https://github.com/jselsing}{GitHub profile})}


\section*{Teaching}

\begin{itemize}
	
	\item \emph{Introductory Cosmology}, TA, 2013, University of Copenhagen
    \item \emph{Introductory Classical Mechanics}, TA, 2013, University of Copenhagen
	\item \emph{Advanced Classical Mechanics}, TA, 2013, University of Copenhagen
\end{itemize}


\section*{Public outreach}

\begin{itemize}
	\item \emph{Gravitational waves, Kilonovae and the history of gold.}, 2018, public outreach talk, Rysensteen Gymnasium, DK
	\item \emph{Gravitational waves, Kilonovae and the history of gold.}, 2018, public outreach talk, naNODE, DTU, DK
    \item \emph{Science breakthrough of the year 2017 (\AA rets videnskabelige gennembrud)}, 2018, DR2 Dagen - national danish television, DK	
    \item \emph{Gravitational Waves and their EM counterparts}, 2017, public outreach talk, Brorfelde observatory, DK
    \item \emph{Gravitational waves, Kilonovae and the history of gold.}, 2017, Danish society of engineers, IDA, DK
    \item \emph{Gravitational waves, Kilonovae and lots of gold.}, 2017, motivational day for interested elementary school students, University of Copenhagen, DK
    \item \emph{Gravitational waves, Kilonovae and the origin of gold.}, 2017, theme day for high-school teachers, University of Copenhagen, DK
    \item \emph{Gravitational waves from a neutron star merger}, 2017, high school motivational talk, University of Copenhagen, DK
    \item \emph{GW170817/GRB170817A}, 2017, Appeared on danish national television and in most major newspapers, cited in 52 individual articles on the observations of GW170817/GRB170817A, DK    
    \item \emph{GRBs and Supernovae}, 2016, public outreach talk, NOVA astronomical society, Helsingor, DK
    \item \emph{Astronomy; how to be cool}, 2016, motivational talk for young students, University of Copenhagen, DK
    
    \item Guide, 2010-2012, Tycho Brahe Planetarium, DK

\end{itemize}

\end{document}
